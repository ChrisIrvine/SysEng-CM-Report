% BOF Preamble
\documentclass[cmpstyle]{ueacmpstyle}
% imports
\usepackage{fancyhdr}
\usepackage{csquotes}
\pagestyle{fancy}
\fancyhead{}
\fancyfoot{}
\lfoot{100036248}
\rfoot{Page \thepage}
\renewcommand{\footrulewidth}{0.4pt}
% macros
% EOF Preamble

% BOF Document
\begin{document}
	\title{Much ado about Everything: \\ The Configuration Management Story}
	\author{Christopher A. Irvine}
	\date{\today}
	\maketitle
	
	\begin{abstract}
		200 words of text here about how awesome this is
	\end{abstract}

	\section{Introduction} \label{sec:intro}
	Configuration Management (CM) has helped many organisations across the globe manage the development and maintenance of complicated systems. However, the influence of CM does not stop in the office. The techniques used to incorporate the CM functions into a project have seeped out into every day life. In a similar manner, CM has taken many every day organisational techniques to heart. 
	
	In this paper we will look at where CM originated from and how it has evolved to be relevant in the industries of today (see Section \ref{sec:history}). Then we will look how the functions of CM (see Section \ref{sec:principles}) are used both in the office and at the home (see Section \ref{sec:using}). Before performing a critical analysis on the benefits gained from incorporating CM into a project or personal life, to try to justify the costs associated with that incorporation (see Section \ref{sec:embracing}). 
	
		\subsection{Defining Configuration Management (CM)} \label{sec:definition}
		Before we continue to answer the questions outlined in Section \ref{sec:intro}, we should have an understanding of what the modern definition of CM is. 
		
		According to the \emph{Association of Project Management (APM)}: 
		
		\begin{quote}
		``Configuration Management encompasses the administrative activities concerned with the creation, maintenance, controlled change and quality control of the scope of work." (\cite{apmDef})
		\end{quote}
		
		Expanding upon this definition; CM is a collection of principles, techniques and characteristics that aim to control the execution of a project. This allows for the Project Manager to ensure that all work regarding the project is of a high quality by deploying CM techniques, ensuring that short-term targets and long-term goals are achieved.
	
	\section{History of CM} \label{sec:history}
	In this section we will take a brief look at where CM came from and why it was created (see Section \ref{sec:origin}), and how CM spread across the globe (see Section \ref{sec:adoption}). Finally we will learn how the CM has evolved to remain relevant in the IT industry, and what standards have emerged as a result.
	
		\subsection{Origins} \label{sec:origin}
		The origins of CM can be traced back to the U.S. Department of Defence (DoD), where the need for universal hardware standards was required in order to make maintenance of the equipment manageable. It was not until the 1960s that CM became a technical discipline of its own, when the DoD released a series of military standards known as the \emph{``480 series"}. These standards were regularly updated and eventually consolidated into MIL-HDBK-61 in 1991, which contained a series of technical standards supported by standards developing organisations (SDO)(personal communication, \cite{dod-history}) \footnote{A Standards Developing Organisation (SDO) is a body whose primary activities revolve around the improvement and development of technical standards within a given field (\cite{history-standards}).}.
		
		\subsection{Adoption into Industry} \label{sec:adoption}
		SDOs regulated their own industries through a collection of standards publications starting in the late 80s and early 90s. These various issues have evolved into a widely distributed an accepted standard on CM known as \emph{ANSI-EIA-649-1998 (EIA-649)}, a venture helmed by the Electronic Industries Alliance (EIA). EIA-649 has provided the base for many specialised CM techniques since the 90s, but this document describes the five primary functions of CM (\cite{EIA-649}). 
		
		\subsection{Software Configuration Management (SCM)} \label{sec:scm}
		
	\section{Functions of CM/SCM} \label{sec:principles}
	
	\section{Using CM} \label{sec:using}
	
		\subsection{In the Office} \label{sec:office}
		
		\subsection{At Home} \label{sec:home}
		
	\section{Embracing CM} \label{sec:embracing}
	
		\subsection{Benefits of CM} \label{sec:benefits}
		
		\subsection{Costs} \label{sec:costs}
		
	\section{Conclusion} \label{sec:conc}
	
	\bibliographystyle{apalike}
	\bibliography{report}
\end{document}

% EOF Document